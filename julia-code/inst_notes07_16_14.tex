Instantons under AC Power Flows
=========

Abstract
---------

The prevalence and variability of renewables in modern power networks has researchers and system operators asking: what happens when the wind or sun changes, and could this harm the grid? The _instanton problem_, an active LANL research direction, addresses this question. The instanton is the most likely renewable generation pattern that would result in a violation of at least one grid constraint; it lies on the exterior of the network's nonlinear parameter space. By using the linear DC approximation, researchers have shown that the instanton may be found analytically, albeit with a degree of inaccuracy. The new method, which relies on fast-decoupled power flow, improves accuracy by incorporating voltage magnitudes while maintaining the vital convergence properties of the DC instanton problem. Results are shown for a modified version of the IEEE RTS-96 network.

Introduction
-------

Consider a transmission or sub-transmission system with a combination of conventional (e.g. steam turbine) and wind generation. This system may be described by the graph triple ${\cal G}=({\cal V},{\cal E},{Z})$, where the arguments stand for the set of vertices (nodes), set of edges (lines) and set of edge impedances, respectively. To distinguish various types of nodes, we let ${\cal V}_d$ denote the set of demand-only nodes and ${\cal V}_R$ denote the set of nodes with wind generation. The Power Flow (PF) equations are:

$$\begin{align}
\forall i\in{\cal V}:&\quad G_{p,i} + R_{p,i} - D_{p,i} = \sum_{k:(i,k)\in{\cal E}} P_{ik},\quad
P_{ik}\equiv
\mbox{Re}\left(V_i\frac{V_i^*-V_k^*}{z_{ik}^*}\right) \\
\forall i\in{\cal V}_{pq}:&\quad G_{q,i} + R_{q,i} - D_{q,i} =\sum_{k:(i,k)\in{\cal E}} Q_{ik},\quad Q_{ik}\equiv\mbox{Im}\left(V_i\frac{V_i^*-V_k^*}{z_{ik}^*}\right) \\
\forall i\in{\cal V}_{pv}:&\quad |V_i|=1,
\end{align}$$

where ${\cal V}_{pq}\subset {\cal V}$ stands for the set of $pq$ nodes and ${\cal V}_{pv}\subset {\cal V}$ stands for the set of $pv$ nodes. Each $pq$ node has its active ($p$) and reactive ($q$) power specified, and each $pv$ node has its active power ($p$) and voltage magnitude ($v$) specified. In physical terms, all nodes without generation (load only) are $pq$ nodes, while nodes with generation --both conventional (e.g. steam turbines) and wind-- are $pv$ nodes. $V=(|V_i|\angle \theta_i,~i\in{\cal V})$ is the vector of complex voltages, and $z_{ik} = r_{ik}+{\it i} x_{ik}$ is the complex impedance of the line between nodes $i$ and $k$ consisting of resistance $r_{ik}$ and reactance $x_{ik}$.

The left-hand sides of the first two equations above are injection terms: they sum conventional ($G$), wind ($R$), and demand ($D$) components to obtain total injection for each node $i$. We will assume that within the time span of interest $D_{p}$ and $D_{q}$ remain constant, though these assumptions may be altered to account for load variations, frequency demand response, or other effects of interest.

The power output of a wind farm is governed by a Gaussian distribution:
$$\begin{align}
R_p\sim \exp\left(-{\cal S}(R)\right),\quad {\cal S}(R_p)=\frac{1}{2}\sum_{i,j\in{\cal V}_R}(R_{p,i}-\bar{R}_{p,i})(\Lambda^{-1})_{ij}(R_{p,i}-\bar{R}_{p,i})
\end{align}$$

where $\bar{R}_p =(\bar{R}_{p,i}|i\in{\cal V}_R)$ is the vector of the average/forecast wind and $\Lambda$ is the covariance matrix representing correlation between wind sites.  For time spans under an hour and wind farms 100 miles or further apart, wind does not travel fast enough to introduce significant correlation. Under these conditions, one can safely set $\Lambda = I$.

Conventional generation ($G_{p}=(G_{p,i}|i\in{\cal V}_G\subset{\cal V})$) varies to take up any mismatch between total demand and total generation. The mechanism by which conventional generators achieve this balance is commonly referred to as ``droop control'' and depends on a parameter $\alpha$ and a distribution vector $d$ with one distribution coefficient for each generator.

Each line in a power grid has a power flow limit, and each nodal voltage must remain near its nominal magnitude. Safe operation of the power grid, therefore, means the following conditions must be met:
$$\begin{align}
\forall (i,k)\in{\cal E}_*:&\quad |P_{ik}|\leq \bar{P}_{ik}\\
\forall i\in{\cal V}_{pq}:&\quad ||V_i|-1|\leq \Delta_i
\end{align}$$
The first equation places limits on the active power flowing on each line, while the second ensures nodal voltages do not deviate too far from their nominal values of 1 pu. Let us enumerate all inequalities in the two above equations by $m=1,\cdots M$, and introduce the set of $M$ additional conditions where one of the inequalities turns into equality. We will denote this condition by ${Equality}(m)$.

With all necessary governing equations and constraints specified, we are ready to introduce the AC instanton problem. The goal of instanton analysis is to find the most probable wind pattern $R_p$ that leads to violation of the network conditions. This problem is formally stated as follows:
$$\begin{align}
\min_{m\in [1:1:M]}{\cal S}(R_p^*)&\\
 m\in [1:1:M]:\quad R_{p,m}^* &=\left.\underset{R}{\mbox{argmin}} ~\min_V {\cal S}(R)\right|
\end{align}$$
Subject to all network constraints.

Note that each $R{p,m}^*$ is the most likely wind generation pattern that saturates constraint $m$; thus, the most likely of all $R_{p,m}^*$ patterns is the instanton.

\section{DC Approximation: Line Flows}

As with all power flow analysis, the instanton problem becomes dramatically simpler with the [DC approximation][1]. The DC version of the previously defined instanton problem, known from now on as the "DC instanton problem", has been studied in at least three previous works.

* In [here][2], the DC instanton problem was posed with load uncertainty (governed by probability distribution) and the option of instantaneous adjustment of controllable generation to respond to the effects of wind variation.

* In [here][3], a more realistic (and currently discussed) setting was presented, where all controllable generators respond in an affine way to wind variability. 

* Finally, [this paper][4] addresses the simplest interesting version of the DC instanton problem, where one generator takes up all slack and only line constraints are considered. 

The instanton problem formulations in [this paper][4] and [this paper][3] have analytic solutions. For completeness we will reproduce these results below.

{\color{red} Begin Jonas addition}

Consider first the DC instanton problem with one slack bus as analyzed in [4]. Suppose the network has $n$ nodes and $m$ wind farms. After picking a node with a conventional generator to be the slack bus, there are $m$ wind nodes and ($n-1-m$) conventional generation/load nodes. Phase angles are mapped to power injections via a matrix equation, 
$$\begin{align}
\mathbf{B}\mathbf{\theta} = \mathbf{P}~,
\end{align}$$
where $\mathbf{B}$ is the square, symmetric admittance matrix whose elements are as follows:
$$\begin{align}
B_{ik} &= \begin{cases}
0, & (i,k) \notin {\cal E} \\
-1/x_{ik}, & (i,k) \in {\cal E} \\
\sum\limits_{k:(i,k) \in {\cal E}}1/x_{ik}, & i = k
\end{cases}~.
\end{align}$$
Because wind generation must be treated separately from conventional generation, we rearrange rows of the admittance matrix and partition it as follows:
$$\begin{align}
\begin{bmatrix}
B_R \\
B_b
\end{bmatrix} \mathbf{\theta} - \begin{bmatrix}
R \\
b
\end{bmatrix} &= 0~,
\end{align}$$
where $B_R$ and $B_b$ are rows of the admittance matrix corresponding to wind and conventional nodes, respectively, and $R \in \mathbb{R}^m$ and $b \in \mathbb{R}^{n-1-m}$ are vectors of wind and conventional generation respectively. To force an individual line flow to its limit $\bar{p}_{ik}$, we impose the following constraint:
$$\begin{align}
(\theta_i - \theta_k)/x_{ik} - \bar{p}_{ik} &= 0 \\
\nonumber \implies s_{ik}^\top \theta - \bar{p}_{ik} &= 0~,
\end{align}$$
where $s_i$ is a column vector with $1/x_{ik}$ in the $i^\text{th}$ position and $-1/x_{ik}$ in the $k^\text{th}$ position. The final constraint ensures wind generation limits are respected:
$$\begin{align}
 0 \leq R_p \leq R_\text{max}
\end{align}$$

If the base-case wind generation is denoted by $\bar{R}_p$, then the wind generation vector $R_{p,ij}$ that is closest to $\bar{R}_p$ and causes line $(i,j)$ to saturate is expressed as
$$\begin{align}
R_{p,ij} = \underset{R_p}{\text{argmin}}~\frac{1}{2}(R_p-\bar{R}_p)^\top & \Lambda(R_p-\bar{R}_p)~, 
\end{align}$$
subject to the power balance constraints, active flow constraint, and renewable output constraints. This equation is solved for each line using Lagrange multipliers. The Lagrangian is
$$\begin{align}
\mathcal{L} \left(R ,\theta ,{\lambda }_{R },{\lambda }_{b},\gamma \right) &= \frac{1}{2}{\left(R -{R }_{0}\right)}^{\top }\Lambda \left(R -{R }_{0}\right)+{\lambda }_{R }^{\top }\left({B}_{R }\theta - R\right) + \lambda_b^\top (B_b\theta - b) + \gamma(s_{ij}^\top \theta - \bar{p}_{ij})~,
\end{align}$$
and the [Karush-Kuhn-Tucker conditions][1] yield the following set of linear equations:
$$\begin{align}
\frac{\partial \mathcal{L}}{\partial R} = 0 &= (R - R_0)^\top \Lambda - \lambda_R^\top \\
\frac{\partial \mathcal{L}}{\partial \theta} = 0 &= \begin{bmatrix}
\lambda_R^\top & \lambda_b^\top
\end{bmatrix} \begin{bmatrix}
B_R \\
B_b
\end{bmatrix} + \gamma s_{ik}^\top \\
\frac{\partial \mathcal{L}}{\partial \lambda_R} = 0 &= B_R \theta - R \\
\frac{\partial \mathcal{L}}{\partial \lambda_b} = 0 &= B_b\theta - b \\
\frac{\partial \mathcal{L}}{\partial \gamma} = 0 &= s_{ij}^\top \theta - \bar{p}_{ij}
\end{align}$$
Thus, the wind generation pattern corresponding to saturation of the line between nodes $i$ and $k$ is represented by $R$ in the following matrix equation:
$$\begin{align}
\begin{bmatrix}
\Lambda & 0 & -I & 0 & 0 \\
0 & 0 & B_R^\top & B_b^\top & s_{ik} \\
-I & B_R & 0 & 0 & 0 \\
0 & B_b & 0 & 0 & 0 \\
0 & s_{ik}^\top & 0 & 0 & 0
\end{bmatrix} \begin{bmatrix}
R \\
\theta \\
\lambda_R \\
\lambda_b \\
\gamma
\end{bmatrix} &= \begin{bmatrix}
\Lambda R_0 \\
0 \\
0 \\
b \\
\bar{p}_{ik}
\end{bmatrix}
\end{align}$$

Solving the KKT conditions for each line in the network yields a set of wind generation patterns $P=\{\bar{R}_{p,ik}|(i,k) \in {\cal E}$ \}, and we can now obtain the instanton $R_p^*$ as the most likely member of $P$ (closest to $\bar{R}_p$):
$$\begin{align}
R_p^* &= \underset{R_p \in P}{\text{argmin}}~\frac{1}{2}(R_p - \bar{R}_p)^\top \Lambda (R_p - \bar{R}_p)
\end{align}$$
Because $R_p^*$ is a member of $P$, it is guaranteed to saturate at least one line. Thus, this DC instanton problem formulation yields the solution to the DC instanton problem with one slack bus. The numerical example used in \cite{12BS} is a portion of the DTE 40 kV sub-transmission network.

Now consider the DC instanton problem with distributed slack, as analyzed in \cite{11CSPB}. In physical terms, distributed slack means that all conventional generators respond to variation in wind generation simultaneously, each in proportion to its base case output (there must still be one node with phase angle defined to be zero). This behavior is modeled by a scaling parameter $\alpha$:
$$\begin{align}
 \alpha := \frac{\sum_{i \in {\cal G}_d}d_i - \sum_{i \in {\cal G}_R}R_i}{\sum_{i \in {\cal G}_b}b_i}~,
\end{align}$$
where ${\cal G}_d$ is the set of demand-only nodes, ${\cal G}_R$ is the set of nodes with wind generation, and ${\cal G}_b$ is the set of nodes with conventional generation. In physical terms, Eq. (\ref{alpha}) states that the product of $\alpha$ and total conventional generation must be equal to total net demand (demand less wind generation)\footnote{Intuitively, then, distributed slack means that all generators scale their output by $\alpha$ to compensate for changes in net demand caused by wind variation.}. Thus, each conventional generator's output $b_i$ is replaced by $\alpha b_i$, where $\alpha = 1$ in the base case. The power condition of Eq. (\ref{baghsorkhi11}) is replaced by
$$\begin{align}
 \underline{\alpha} \leq \alpha \leq \bar{\alpha}~,
\end{align}$$
where $\underline{\alpha}$ comes from the generator that is least able to reduce its production and $\bar{\alpha}$ comes from the generator that is least able to increase its production. If $\mathbf{\theta}$ in Eq. (\ref{DC_map}) and $\alpha$ in Eq. (\ref{chertkov7}) are expressed in terms of generation, we obtain the following set of $K$ inequality conditions for the system:
$$\begin{align}
 \forall (i,k) \in {\cal E}:~|(\tilde{B}P_i) - (\tilde{B}P_k)| \leq x_{ik}\bar{p}_{ik}~; \\
\sum\limits_{i \in {\cal G}_d} d_i - \underline{\alpha} \sum\limits_{i \in {\cal G}_b} b_i \geq \sum\limits_{i \in {\cal G}_R} R_i \geq \sum\limits_{i \in {\cal G}_d} d_i - \bar{\alpha} \sum\limits_{i \in {\cal G}_b} b_i~,
\end{align}$$
where $\tilde{B}$ is the quasi-inverse of the admittance matrix (accounting for the phase angle of the slack bus being zero). Eqs. (\ref{chertkov13}-\ref{chertkov14}) reduce the DC instanton problem in Eq. (\ref{baghsorkhi8}) to a tractable set of constraints\footnote{The set of $K$ constraints may be viewed as a \textit{tractable polytope} whose interior contains all safe operating points of the network. See \cite{11CSPB}.}. To solve for the instanton, one need only form the set ${\cal M}$:
$$\begin{align}
 \min_{i=1,\ldots,K}{\cal M}_i,\quad {\cal M}_i = \min_{R} S(R)|_{\underline{c}_i \cup (Eqs. (\ref{chertkov13}-\ref{chertkov14}) \backslash c_i)}~,
\end{align}$$
where $K$ is the number of constraints in Eqs. (\ref{chertkov13}-\ref{chertkov14}), $c_i$ is the $i^\text{th}$ inequality constraint, and $\underline{c}_i$ is $c_i$ with inequality replaced by equality. As with the single slack bus formulation from \cite{12BS}, Eq. (\ref{chertkov15}) may be solved quickly, even for large systems. The numerical example in \cite{11CSPB} is a modified version of the IEEE RTS-96 network.


Nonlinear Decoupled Power Flow Formulation
----------

The Fast-Decoupled Power Flow method described in Section \ref{sec:FDPF} may be modified to solve the instanton problem. Suppose we use the FDPF to obtain an operating point for the system. We would like to find the renewable generation pattern $\bar{R}_{ik}$ that would result in saturation of line $(i,k)$. The remainder of the analysis follows the line of reasoning in Section 4 Part C of \cite{12BS}.

To find the renewable generation pattern most likely to saturate line $(i,k)$, we seek a linear mapping between $\Delta P_{ik}$ (scalar deviation in active flow on line $(i,k)$) and $\Delta R$ (vector of deviations in renewable generation). Once we find this mapping, we can perform constrained quadratic optimization to find the instanton. Recall that active power flow on a line is a function of four state variables,
$$\begin{align}
P_{ik} &= f(\theta_i,\theta_k,V_i,V_k)~,
\end{align}$$
which may be linearized to map small perturbations in arguments to differences in line flow:
$$\begin{align}

\Delta P_{ik} &= \begin{bmatrix}
\frac{\partial f}{\partial \theta_i} & \frac{\partial f}{\partial \theta_k} & \frac{\partial f}{\partial V_i} & \frac{\partial f}{\partial V_k}
\end{bmatrix} \begin{bmatrix}
\Delta \theta_i \\
\Delta \theta_k \\
\Delta V_i \\
\Delta V_k
\end{bmatrix}~,
\end{align}$$
where the partial derivatives are evaluated at the system operating point. One assumption of FDPF is that voltage magnitudes have a negligible effect on active power flow, e.g. $\frac{\partial f}{\partial V_i} = 0$ for any voltage $V_i$. This simplifies Eq. \ref{dP_full_mapping} to
$$\begin{align}

\Delta P_{ik} &= \begin{bmatrix}
\frac{\partial f}{\partial \theta_i} & \frac{\partial f}{\partial \theta_k}
\end{bmatrix} \begin{bmatrix}
\Delta \theta_i \\
\Delta \theta_k
\end{bmatrix}
\end{align}$$
From the FDPF formulation (See Section \ref{sec:FDPF}), and specifically Eq. \ref{FDPF_dt}, we obtain the following linear relationship between $\Delta \theta$ and $\Delta P/V$:
$$\begin{align}

\begin{bmatrix}\vspace{6pt}
\Delta \theta_i \\
\Delta \theta_k
\end{bmatrix} &= \begin{bmatrix}\vspace{6pt}
B'^{-1}_{i,R} \\
B'^{-1}_{k,R}
\end{bmatrix} \begin{bmatrix}
\Delta R/V
\end{bmatrix}~,
\end{align}$$
where $B'^{-1}_{i,R}$ refers to the $i^\text{th}$ row of $B'^{-1}$, with only columns corresponding to renewable generation nodes included. Substituting Eq. \ref{dt_to_dP} into Eq. \ref{dP_theta_mapping} yields
$$\begin{align}

\Delta P_{ik} &= \underbrace{\begin{bmatrix}\vspace{6pt}
\frac{\partial f}{\partial \theta_i} & \frac{\partial f}{\partial \theta_k}
\end{bmatrix} \begin{bmatrix}\vspace{6pt}
B'^{-1}_{i,R} \\
B'^{-1}_{k,R}
\end{bmatrix}}_{S_{I,R}} \begin{bmatrix}
\Delta R/V
\end{bmatrix}~,
\end{align}$$
which approximately maps changes in renewable generation to changes in active power flow on line $(i,k)$. With Eq. \ref{dP_to_drho} defined, we may now perform constrained quadratic optimization to find the most likely renewable generation pattern that saturates line $(i,k)$:
$$\begin{align}
\Delta \bar{R}_{ik} &= \text{argmin}_{\Delta R} \frac{1}{2}\Delta R^\top W\Delta R \\
\nonumber \text{subject to:} \\
P_{ik}^\text{lim} - P_{ik}^0 &= \Delta P_{ik} = S_{i,R} \begin{bmatrix}
\Delta R/V
\end{bmatrix}
\end{align}$$
The Lagrangian for this problem is
$$\begin{align}

{\cal L}(\delta R,\lambda) &= \frac{1}{2}\DeltaR^\top W \Delta R + \lambda(S_{I,R}\Delta R - \delta P_{ik})~, 
\end{align}$$
and the Karush-Kuhn-Tucker conditions yield
$$\begin{align}

\begin{cases}
\frac{\partial {\cal L}}{\partial \Delta R} &= 0 = \Delta R^\top W + \lambda S_{I,R}~, \\
\frac{\partial {\cal L}}{\partial \lambda} &= 0 = S_{I,R}\Delta R - \Delta P_{ik}~.
\end{cases}
\end{align}$$
When Eqs. \ref{FDPF_KKT} are arranged in matrix form, we have
$$\begin{align}

\begin{bmatrix}\vspace{6pt}
W & S_{I,R}^\top \\
S_{I,R} & 0
\end{bmatrix} \begin{bmatrix}\vspace{6pt}
\frac{\Delta R}{V} \\
\lambda
\end{bmatrix} &= \begin{bmatrix}\vspace{6pt}
0 \\
\Delta P_{ik}
\end{bmatrix}~,
\end{align}$$
which may be solved for $\bar{\Delta R}$, the vector of changes in renewable generation that is most likely to saturate line $(i,k)$. Unlike the formulation in \ref{sec:inst-DC}, the optimization output is a vector of changes in renewable generation rather than the renewable generation pattern itself. Thus, the solution of \ref{FDPF_matrix} is an update vector that must be added to base-case renewable generation. This leads to the following FDPF instanton solution method:
\begin{framed}
\begin{enumerate}
	\item Select a line.
	\begin{enumerate}
		\item Solve the FDPF to obtain an operating point for the network.
		\item Calculate the sensitivity matrix $S_{I,R}$ for the chosen line.
		\item Solve Eq. \ref{FDPF_matrix} for $\bar{\Delta R}$.
		\item Use $\bar{\Delta R}$ to update renewable generation.
		\item Repeat 1-3 until $\bar{\Delta R}$ becomes acceptably small. Denote the final renewable generation pattern $R^*_{ik}$.
	\end{enumerate}
	\item Repeat Step 1 for each line in the system.
	\item Determine which $R^*_{ik}$ is closest to base-case renewable generation (in a 2-norm sense, weighted by $W$). This is the instanton.
\end{enumerate}

\end{framed}
Like the full non-linear AC instanton problem, the FDPF instanton problem does not guarantee convergence, and even if convergence occurs, the solution may not be unique.

{\color{red} End Jonas addition}

\section{Instanton in the nonlinear DC approximation}
inst-nlDC}

%{\color{red}  ... placeholder for description of how the nonlinear DC version of Eq.~(\ref{inst}) is transformed under the nonlinear DC approximation explained in Appendix \ref{subsec:nonlinear-DC} ... then, acting in the spirit of what is explained in Section III of \cite{13BBC} one may hope to turn Eq.~(\ref{inst}) into a convex optimization ...}

Let us show how the nonlinear DC version of Eq.~(\ref{inst}), explained in Appendix \ref{subsec:nonlinear-DC}, allows to restate the general instanton problem for one of the facets of the ``error surface" (\ref{inst_m}) as a convex optimization problem. Disclaimer -- we simply follow in here the approach utilized in Section III of \cite{13BBC} to derive a convex formulation of the OPF in the nonlinear DC approximation.

Considered within the nonlinear DC approximation the instanton optimization problem (\ref{inst_m}) for a facet of the error surface associated with the overload of a line $(k,l)$ in the $k\to l$ direction becomes
\begin{equation}
\left.\min_{\theta,w} {\cal S}(w)\right|_{
\begin{array}{cc}
p_i-d_i+w_i=\sum_{j:(i,j)\in{\cal E}}\beta_{ij}\sin(\theta_i-\theta_j), & \forall i\in{\cal V}\\
\beta_{kl}\sin(\theta_k-\theta_l)=\bar{p}_{kl} &
\end{array}
}.
inst1}
\end{equation}

Following Section III of \cite{13BBC} we replace Eq.~(\ref{kl-inst1}) by
\begin{eqnarray}
\left.\min_{R,\delta,w} \left({\cal S}(w)+D\sum_{(i,j)\in{cal E}}\left(\beta_{ij}\psi(R_{ij})-\phi\log \delta_{ij}\right)\right)\right|_{
\begin{array}{cc}
\sum_{j:(i,j)\in{\cal E}}\beta_{ij}R_{ij}=p_i-d_i+w_i, & \forall i\in{\cal V}\\
R_{ij}\leq 1-\delta_{ij},\quad \delta_{ij}\geq 1, & \forall (i,j)\in{\cal E}\\
\beta_{kl}R_{kl}=\bar{p}_{kl} &
\end{array}
}.
inst2}
\end{eqnarray}
%where the positive coefficients $D$ and $\phi$ are chosen sufficiently large and small respectively.
It is direct to check that the optimization problem (\ref{kl-inst2}) is convex.

The result proven in \cite{13BBC}, and extended here to the instanton optimization, states (informally) that by choosing the positive coefficients $D$ and $\phi$ to be, respectively, sufficiently small and sufficiently large, one guarantees that Eq.~(\ref{kl-inst2}) provides an $\epsilon$-approximation for Eq.~(\ref{kl-inst1}), where $\epsilon\to 0$ when $D\to\infty$ and $\phi\to 0$.





\section{Convexification of the general AC-instanton}

{\color{red} ... this is a placeholder for the yet to be developed approximation, with the main ideas from \cite{14HCV}, applied to Eq.~(\ref{inst}) ... Pascal -- this part is left for your team ...}

\appendix

\section{Power Flow Approximations}
PF_appr}

\subsection{DC Approximation: Active Power}
DC}

DC approximation assumes that
\begin{itemize}
\item lines are lossless, i.e.
\begin{equation}
\forall (i,j)\in{\cal E}:\quad r_{ij}=0;
\end{equation}
\item voltages are constant, i.e. in per unit,
\begin{equation}
\forall i\in{\cal V}:\ |V_i|=1;
\end{equation}
\item phase differences over all lines are small,
\begin{equation}
\forall (i,j)\in{\cal E}:\ \sin(\theta_i - \theta_j) \approxeq (\theta_i-\theta_j)~. 
\end{equation}
\end{itemize}
These three approximations greatly simplify power flow analysis, reducing the power flow equations in Eq. (\ref{PF_p}) to
the PF Eqs.~(\ref{PF_p}) turn into
\begin{eqnarray}
&& \forall i\in{\cal V}:\quad p_i+w_i-d_i=\sum_{j:(i,j)\in{\cal E}} \frac{\theta_i-\theta_j}{x_{ij}}.

\end{eqnarray}
This set of linear equations is sufficient to resolve phases and compute approximate active power flows over lines, $p_{i\to j}\approx (\theta_i-\theta_j)/x_{ij}$.
%(The DC version of Eqs.~(\ref{PF_q_pq}) only needs to be accounted for if one is interested to reconstruct reactive powers needed to maintain voltages constant all over the network.)
Here in Eq.~(\ref{PF_p_DC}) only the leading terms in the following three small parameters --- zero order in $r_{ij}/x_{ij}$, zero order in $1-v_i$, and first order in $(\theta_i-\theta_j)$ -- are kept. Eqs.~(\ref{PF_p_DC}) can also be restated in vector notation:
% I have never seen this formulation before...

\begin{eqnarray}
p+w-d=\left(\nabla^T\beta\nabla \right)\theta,vector}
\end{eqnarray}
where $\nabla$ is the $|{\cal E}|\times|{\cal V}|$-dimensional matrix of discrete graph differentials (i.e. $\nabla\theta$ is the $|{\cal E}|$-dimensional vector of phase differences) and $\beta$ is the diagonal $|{\cal E}|\times|{\cal E}|$-dimensional matrix with $\beta_{ij}=1/x_{ij}$ placed at the respective diagonal elements.


\subsection{DC Approximation: Voltage Magnitudes}
FDPF}
The FDDC approximation improves on the bare DC approximation by accounting for the first order in $\varepsilon_i=1-v_i$ corrections within the reactive power balance. In other words, one relaxes Eqs.~(\ref{v_unity}) at the pq-nodes in the approximations behind the bare DC by
\begin{equation}
\forall i\in{\cal V}_{pq}:\quad 1-|V_i|=\varepsilon_i\ll 1,
\end{equation}
Then, keeping in Eqs.~(\ref{PF_q_pq},\ref{PF_q_pv}) only the first order terms in $\varepsilon$ one derives
\begin{eqnarray}
&& \forall i\in{\cal V}_{pq}:\quad -q_i=\sum_{j:(i,j)\in{\cal E}}
\frac{\varepsilon_i-\varepsilon_j}{x_{ij}},\\
&& \forall i\in{\cal V}_{pv}:\quad \varepsilon_i=0.
\end{eqnarray}
This formulation is similar to the ``Cold-Start LPAC Model'' in \cite{coffrin_unpublished}.

\subsection{Fast Decoupled DC approximation}

{\color{red}Begin Jonas addition}

Unlike the bare DC approximation, the FDDC approximation cannot be solved by matrix inversion; an iterative scheme must be used instead. To see why this is, begin with the nonlinear functions describing power injection at each node:
$$\begin{align}

\begin{bmatrix}
P_i \\
Q_i \end{bmatrix}  &= \begin{bmatrix}
f_i(\theta,V) \\
g_i(\theta,V)
\end{bmatrix}
\end{align}$$

The Newton-Raphson method relies on a first-order approximation of Eqns. \ref{NL_pf}, mapping small changes in power to small changes in state as follows:
$$\begin{align}

\begin{bmatrix}
\vspace{6pt}
\Delta P_i \\
\Delta Q_i \end{bmatrix} &= \begin{bmatrix}\vspace{6pt}
\frac{\partial f_i}{\partial \theta} &
\frac{\partial f_i}{\partial V} \\
\frac{\partial g_i}{\partial \theta} &
\frac{\partial g_i}{\partial V}
\end{bmatrix}
\begin{bmatrix}
\vspace{6pt}\Delta \theta \\
\Delta V
\end{bmatrix}
\end{align}$$
The FDDC method uses Eqns \ref{NL_update} as a starting point for a series of simplifications, the first two of which decouple the active and reactive relationships:
\begin{itemize}
	\item Since active power is influenced primarily by angle differences, approximate $\frac{\partial f_i}{\partial V}$ by zero.
	\item Because reactive power is primarily a function of voltage magnitude differences, approximate $\frac{\partial g_i}{\partial \theta}$ by zero.
\end{itemize}
This leads to the decoupled equations
$$\begin{align}
\Delta P_i &= \frac{\partial f_i}{\partial \theta}\Delta \theta~, \\
\Delta Q_i &= \frac{\partial g_i}{\partial V}\Delta V~.
\end{align}$$
The goal now is to incorporate these two mappings in an iterative scheme to reach a power flow solution. Unfortunately, the partial derivatives are functions of all state variables, so their values will change at each iteration. If we normalize $\Delta P_i$ and $\Delta Q_i$ by voltage magnitude $V_i$ at each node, we can replace the two matrices of partial derivatives by constant matrices:
\begin{itemize}
	\item Replace the matrix of partial derivatives $\frac{\partial f}{\partial \theta}$ by the constant matrix $B'$ with elements defined as follows:
	\begin{align*}
	\begin{cases}\vspace{6pt}
	B'_{ik} &= -\frac{1}{x_{ik}}\quad \text{for }i \neq k \\
	B'_{ii} &= \sum\limits_{k = 1}^{n}\frac{1}{x_{ik}}
	\end{cases}
	\end{align*} \\
	\item Replace the matrix of partial derivatives $\frac{\partial g}{\partial V}$ by the constant matrix $B'' = B$, where $B$ is taken from the admittance matrix $(Y = G + jB)$.
\end{itemize}
After these simplifications, we obtain the Fast-Decoupled Power Flow method \cite{hiskens2011}:
\begin{framed}
\begin{enumerate}
	\item Build and factorize $B'$ and $B''$.
	\item Initialize voltage magnitudes to 1 pu and voltage angles to 0.
	\item For each iteration:
	\begin{enumerate}
		\item Compute normalized active mismatches:
		$$\begin{align}
		
		\Delta P_i/V_i &= P_i^{sp}/V_i - \sum\limits_{k = 1}^{n}V_k(G_{ik}\cos \theta_{ik} + B_{ik}\sin \theta_{ik})
		\end{align}$$
		\item Solve for $\Delta \theta$:
		$$\begin{align}
		
		\begin{bmatrix}
		\Delta P/V
		\end{bmatrix} &= B'\Delta \theta
		\end{align}$$
		\item Update $\theta$ using $\Delta \theta$.
		\item Compute normalized reactive mismatches:
		$$\begin{align}
		
		\Delta Q_i/V_i &= Q_i^{sp}/V_i - \sum\limits_{k = 1}^{n}V_k(G_{ik}\sin \theta_{ik} - B_{ik}\cos \theta_{ik})
		\end{align}$$
		\item Solve for $\Delta V$:
		$$\begin{align}
		
		\begin{bmatrix}
		\Delta Q/V
		\end{bmatrix} &= B''\Delta V
		\end{align}$$
		\item Update voltage magnitudes using $\Delta V$.
	\end{enumerate}
	\item Continue iterating until all mismatches are acceptably small.
\end{enumerate}
\end{framed}


{\color{red}End Jonas addition}

\subsection{``Nonlinear DC" approximation}
nonlinear-DC}

Here we describe reformulation of the PF Eqs.~(\ref{PF_p},\ref{PF_q_pq}) first discussed in \cite{12BV}. See also \cite{13BBC}.

Consider the following convex optimization problem
\begin{eqnarray}
\min\limits_{R=(R_{ij}, (i,j)\in{\cal E})}&& \sum_{(i,j)\in{\cal E}} \beta_{ij}\psi(R_{ij}),  \\
&\text{s.t.}&  \forall i\in{\cal V}:\quad \sum_{j:(i,j)\in{\cal E}} \beta_{ij} R_{ij}=p_i-d_i+w_i, \\
 &&  \forall (i,j)\in{\cal E}:\quad |R_{ij}|<1,
  \end{eqnarray}
where for $|x| < 1$,
\begin{equation}
\psi(x)\equiv \int_{-1}^x\arcsin(y) dy,
\end{equation}
is a convex function of $x$ since $\arcsin(x)$ is increasing for $x \in [-1, 1]$. Interestingly, in this formulation, if the optimal solution of Eq.~(\ref{IndLoss}) occurs on the boundary of Eq.~(\ref{rho_less_1}), then there exists no feasible solution of the PF Eqs.~(\ref{PF_p},\ref{PF_q_pq}). In other words,  the grid cannot be synchronized.


When optimization problem (\ref{IndLoss}-\ref{rho_less_1}) is well-defined, that is to say, it has an optimal solution which satisfies the strict Eqs.~(\ref{rho_less_1}), we can verify that Eq.~(\ref{IndLoss}) yields Eqs.~(\ref{PF_p},\ref{PF_q_pq}). Let us emphasize that unlike many convex transformations, there is a physical meaning to the optimization in Eq.~(\ref{IndLoss}). It is twice the reactive power losses in all the lines of the network.
(See also Appendix A of \cite{13BBC} for a brief discussion of the duality relation between the optimization formulation just discussed and the so-called energy function approach.)
Notice that the constraints (\ref{conserv}) expresses flow conservation at any node of the network.

Finally, we call this reformulation of the PF equations non-linear DC approximation because of the three conditions stating the bare DC approximation (\ref{r_zero},\ref{v_unity},\ref{theta_small}) we keep in here the first and third ones, but do not enforce the middle condition, thus allowing the phase difference over links to take any value (smaller than $\pi/2$ -- to guarantee feasibility of Eq.~(\ref{IndLoss}), equivalent to synchronization of the grid).

\subsection{Hybrid Approximation -- of Nonlinear DC and Fast Decoupling}
hybrid}

{\color{red} ... it may also be interesting to generalize the nonlinear DC accounting for voltages in the FD-way ... seems straightforward ...}


\bibliographystyle{unsrt}
\bibliography{../Bib/inst,../Bib/inst2}

\end{document}


We now combine the derivation in the prior section with optimal power flow.  Let $C$ be lower bound on the optimal OPF cost\footnote{This can be obtained, for example, by solving a DCOPF problem.}. Let $0 < \epsilon < 1$ be a tolerance, $\beta_{max} = \max_{k} \beta_{k}$, $D = \frac{C \epsilon}{\pi \beta_{max}}$ and $\phi = (-m \log \epsilon)^{-1}$, where $m$ = number of lines. For an arc $k$, let its {\em effective capacity} be $u_k = \min\{1,\capacity_{k}/\beta_{k}\}$. Finally, let the {\em barrier} function $B \, : \, \R_+ \rightarrow \R_+$ be defined by
$$ B(t) \ = \ - \log t.$$
Consider the optimization problem given by
\begin{eqnarray}
&& \min \ f(p) \ + \ D \sum_{ k \in \cL} \susceptance_k
 \left[ \arcsinflow(\lossflow_{k})  \ + \ \phi \, B(\delta_{k}) \right]  \\
&& \text{s.t.} \ \sum_{k \in\cL} \susceptance_{k}a_{ik} \lossflow_{k}=\generation_i-\demand_i+\wind_i   \ \ \ \forall i\in \nodes   \\
&& |\lossflow_{k}|\ + \  u_k \delta_{k} \ \le \ u_k  \\
&& \delta_{k} \ \ge \ 0, \ \ \forall k \in \cL. 
\end{eqnarray}
We have:
\begin{LE}  (a) Suppose $(p^*, \theta^*)$ is the optimizer for the sync-constrained OPF problem. Suppose, further, that for each arc $k = (i,j)$ we have
\begin{eqnarray}
&& | \sin(\theta^*_i - \theta^*_j) | \ \le \ (1 - \epsilon) u_k.
 \end{eqnarray}
Then, defining $R^*_{ij} = \sin(\theta^*_i - \theta^*_j)$
and $\delta^*_{ij} = 1 - \frac{1}{u_{ij}} \, |\lossflow^*_{ij}|$ for each arc
$(i,j)$, we obtain that $p^*, R^*, \delta^*$ is a feasible solution to problem
(\ref{newobj})-(\ref{newlast}) with cost at most
$$ (1 + 2 \epsilon) f(p^*).$$
(b) Conversely, let $(\hat p, \hat R, \hat \delta)$ be an optimal solution to problem (\ref{newobj})-(\ref{newlast}) with objective value $\hat K$, say. Then, with $\hat \theta$ obtained by solving problem (\ref{IndLoss})-(\ref{rho_less_1}) we have that $(\hat p, \hat \theta)$ is feasible for the sync-constrained OPF problem and has cost $f(\hat p) \le \hat K$. (c) Let $(\hat p, \hat R, \hat \delta)$ be as in (b), and suppose that additionally for every arc $k$ we have
\begin{eqnarray}
&& |\hat R_k| \le u_k (1 - \tilde \epsilon), 
\end{eqnarray}
for some value $0 < \tilde \epsilon < 1$.  Let $D \hat \theta$ be the optimal dual variables for constraints (\ref{conserv2}).  Then for every line $k = (i,j)$ we have:
\begin{eqnarray}
&& \hat R_k \ = \ \sin(\hat \theta_i - \hat \theta_j \ + \ \eta_k) 
\end{eqnarray}
where $ | \eta_k | \le (m \, u_k \, \tilde \epsilon \log (1/\epsilon) )^{-1}$.
\end{LE} 